\documentclass{article}
\usepackage{showkeys}
\usepackage{enumerate}
\usepackage{hyperref}
\usepackage{multicol}

% Title page information
\title{\vspace{-4em}\textbf{Computer Science House}\\ Membership Agreement}
\author{}

% Last Modified Date
\newcommand{\datechanged}{Last Updated: \today}
\date{\datechanged}

% Fix margins
\setlength{\evensidemargin}{0in}
\setlength{\oddsidemargin}{0in}
\setlength{\textwidth}{6.5in}
\setlength{\topmargin}{0in}
\setlength{\textheight}{8.5in}

\pagestyle{myheadings}
\markright{{\rm CSH Membership Agreement \hfill \datechanged \hfill Page }}

\begin{document}
\maketitle

\begin{flushleft}
    In the interest of maintaining an open and welcoming community, Computer
    Science House (CSH) expects its members to uphold our organization’s
    values, which are described in this agreement. Furthermore, as an
    organization within RIT, CSH expects its members to abide by the RIT
    Code of Conduct and related policies which can be found at
    \href{https://www.rit.edu/studentaffairs/studentconduct/code-conduct}
    {https://www.rit.edu/studentaffairs/studentconduct/code-conduct}.
\end{flushleft}

\section*{Values}

\paragraph{Inclusion}

As a residential community, CSH has a high standard for inclusivity and
expects its members to regard one another equally and without discrimination
at all times, regardless of actual or perceived:

\begin{multicols}{3}
    \begin{itemize}
        \item age
        \item appearance
        \item bodily or sex characteristics
        \item disability
        \item gender identity or expression
        \item level of experience or education
        \item nationality
        \item race
        \item religion
        \item romantic orientation
        \item sexual identity or orientation
        \item socio-economic status
    \end{itemize}
\end{multicols}

\paragraph{Respect}

As a community of rising adults, CSH expects its members to treat each other
and each other’s ideas with respect and dignity. Constructive criticism is
encouraged and civil discourse is expected, but members are expected to
conduct these types of discussions in a positive and good-faith manner.

\paragraph{Academic Honesty}

As an academic community, CSH cherishes an open exchange of knowledge but
expects its members to maintain academic honesty and to do their own work.

\paragraph{Personal Integrity}

As a student-run organization, CSH expects its members to hold its leadership
(CSH Executive Board) accountable to the decisions they make, and for members
to challenge decisions that run counter to their own personal values.

\section*{Positive Standards}

CSH considers the following to be positive standards of behavior and to
exemplify our values:

\begin{itemize}
    \item{Using welcoming and inclusive language}
    \item{Being respectful of differing viewpoints and experiences}
    \item{Gracefully accepting constructive criticism}
    \item{Focusing on what is best for the community}
    \item{Showing empathy towards other community members}
\end{itemize}

\section*{Negative Standards}

Examples of behaviors that run counter to the values of CSH include but are
not limited to:

\begin{itemize}
    \item{Inappropriate or unwelcome sexual comments, attention or advances}
    \item{Inflammatory behavior, insulting/derogatory comments, and personal
         attacks}
    \item{Public or private harassment}
    \item{Publishing others’ private information, such as a physical or
         electronic address, without explicit permission}
\end{itemize}

\section*{Leadership Expectations}

The Executive Board of CSH (E-board) is expected to clarify
the standards of acceptable behavior as necessary, and to take appropriate
and fair corrective action in response to any instances of unacceptable
behavior. E-board has the right and responsibility to
remove, edit, or reject posts, comments, or other media that is not aligned
to the values in this Membership Agreement from official CSH communication
or collaboration platforms, or to ban temporarily or permanently any member
from these platforms for other behavior that they deem inappropriate,
threatening, offensive, or harmful.

\section*{Leadership Accountability}

There is a path for recourse in the case that a member feels that an E-board
member has abused their position of influence. Examples of behaviors that are
deemed unacceptable for a member of E-board to exhibit include, but are not
limited to:

\begin{itemize}
    \item{Insisting that a member perform an unreasonable action against
         their will}
    \item{Harassing, bullying, or abusing anybody}
    \item{Threatening somebody’s membership to CSH}
    \item{Slandering or spreading rumors}
    \item{Leaking information from closed E-board meetings}
    \item{Other behaviors that violate the values of this Membership Agreement}
\end{itemize}

\begin{flushleft}
Members may report a perceived abuse of an E-board position to the following
people:
\end{flushleft}

\begin{itemize}
    \item{Any other member of E-board}
    \item{The Resident Advisor (RA) of CSH}
    \item{Any non-student res-life employee, via \href{mailto:reslife@rit.edu}
         {reslife@csh.rit.edu} or by calling 585-475-6022}
\end{itemize}

\section*{Scope of Agreement}

This Membership Agreement applies within all CSH spaces,
and it also applies when a member is representing CSH or its community in
public spaces. Examples of representing CSH include using an official CSH
e-mail address, posting via an official social media account, or acting as
an appointed representative at an online or offline event. Representation
of CSH may be further defined and clarified by the CSH E-board.

\section*{Agreement Enforcement}

Instances of abusive, harassing, or other unacceptable behavior may be
reported to the CSH E-board by emailing \href{mailto:eboard@csh.rit.edu}
{eboard@csh.rit.edu}. All complaints will be reviewed and will result in a
response that is deemed necessary and appropriate to the circumstances.
E-board is expected to reasonably protect the privacy of the reporter of an
incident, but may escalate the report to res-life or other RIT faculty at
their discretion. Members who do not follow or uphold this Membership
Agreement in good faith may face temporary or permanent repercussions as
determined by CSH E-board.

\section*{}
\section*{Commitment}
\section*{}

I, \makebox[2.5in]{\hrulefill}, hereby agree to the following conditions for
membership in Computer Science House:

\begin{enumerate}
    \item I will abide by the qualifications and expectations of a member, as
          well as follow the general rules of the House as outlined in the CSH
          Constitution and this Membership Agreement for the entirety of my
          membership.
    \item I will abide by the guidelines and rules set forth in the CSH Code
          of Conduct regarding CSH computer services and member privileges
          for the entirety of my membership.
    \item If I am to fail in the evaluations process, I understand that I will be
          removed from any housing queue or board. If I am already registered to live
          on CSH the following year, I understand my spot will not be revoked.
\end{enumerate}

\begin{flushleft}
I understand that should I violate any of the above conditions, I may be
subject to sanctions as outlined in the CSH Constitution.
\end{flushleft}

\noindent
\begin{tabular}{ll}
\\[8ex]
\makebox[3.5in]{\hrulefill} & \makebox[2.5in]{\hrulefill}\\
Signature & Date\\
\end{tabular}

\end{document}
